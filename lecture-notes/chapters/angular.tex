%!TEX encoding = UTF-8 Unicode
%!TEX root = ../compendium1.tex

%--- Angular -------------------------------------
\chapter{Angular} \label{chapter:angular}

Angular is a framework for single page web applications. An angular application is a set of angular components, and each component have:
\begin{itemize}
\item a tag name, so it can be instantiated in html templates
\item TypeScript code containing state and behaviour
\item a html template, describing how to render the component
\item a styling template describing how to style the component (commonly a css file)
\end{itemize}
Here is an example:
\begin{Code}
import { Component, OnInit } from '@angular/core';

@Component({
  selector: 'app-my-component',
  templateUrl: './my-component.html',
  styleUrls: ['./my-component.scss']
})
export class MyComponent implements OnInit {
  form = {todo: '', done: false};

  constructor(public todoService: TodoService) { }

  ngOnInit() {
  }

  handleSubmit($event) {
    $event.preventDefault();
    console.log('submitting ' + JSON.stringify(this.form));
  }
}
\end{Code}
Angular uses class decorations to add metadata to a component, for example the tag name and the names of the template files.

\section{TypeScript}
Angular is written in TyepScript. TypeScript is a superset of JavaScript, so you can always stick to JavaScript, however there are a few convenient things in TypeScript:
\begin{itemize}
\item Class attributes, in the example above, \code{form} is an attribute in the \code{MyComponent} class. You can add data to your prototypes. The JavaScript way to do the same thing is: 
\begin{Code}
MyComponent.prototype.form = someValue;
\end{Code}
The html template is not part of the class, so you probably want to avoid \code{private} attributes.
\item Look at the code: 
\begin{Code}
constructor(public todoService: TodoService) { }
\end{Code}
It does two things: 1, declares a \code{public todoService: TodoService} property 2, initialises the property to the value passed to the constructor, that is, it is shorthand for:
\begin{Code}
export class MyComponent implements OnInit {
  public todoService: TodoService;
  constructor(todoService: TodoService) { 
    this.todoService = todoService;
  }
}
\end{Code}
The optional typing of TypeScript also enables dependency injection. Angular see that the component needs an instance of the TodoService, and will pass the singleton object of the service to the constructor.
\item Types, TypeScript adds static typing to JavaScript. The type syntax looks like json, for example:
\begin{Code}
public form: {todo: string, done: boolean};
\end{Code}
\end{itemize}
In the course, you should know enough TypeScript to use it in the labs. I will not ask you to write TypeScript code on the exam.

\section{Template Language}
Angular uses html as template language. There are a few additions to the html syntax. Read more about the angular template syntax here: \url{https://angular.io/guide/template-syntax}. Here is an example:
\begin{Code}
<form class="example-form" (ngSubmit)="handleSubmit($event)">
  <mat-form-field class="example-full-width">
    <input matInput name="todo" placeholder="What to do"
               [(ngModel)]="form.todo">
  </mat-form-field>
  <mat-checkbox name="done" [(ngModel)]="form.done" class="example-full-width">
    Check when item is done
  </mat-checkbox>
  <br><button type="submit" mat-raised-button>Submit</button>
</form>
{{form|json}}
\end{Code}
\begin{itemize}
\item The \code{<mat-*>} tags are angular components from the angular material module, see \url{https://material.angular.io/components/form-field/overview}
\item TypeScript expressions: you can embed TypeScript/JavaScript expressions in the html code. Note, the embedded code can not have side effects, so avoid assignments \code{+=, ++} et.c.
\begin{Code}
<h2>{{myTypeScriptFunction()}}</h2>
\end{Code}
A convenient operation that can be used in angular templates is \code{a?.b}, which is equivalent to \code{a ? a.b : a}. It might not look like a big difference, but consider rewriting \code{a?.b?.c?.d?.e} using the \code{ ? : } notation.

The expression context is typically the component instance (name scope). You do not need to write \code{this} to access attributes from the component class.
\item Angular pipes, you can modify the data from an embedded expressions by passing it trough pipes, similar to the pipe in a unix terminal. Pipes can be chained, see \url{https://angular.io/guide/pipes}
\begin{Code}
<h2>{{ birthday | date | uppercase}}</h2>
\end{Code}
There exists selection of predefined pipes which make your life easier. I commonly use \code{json}, which is equivalent to \code{JSON.stringify()}, and \code{async}, which transform an \code{Observable} to the last resolved value, for example:
\begin{Code}
<span>The configuration fetched from the server: 
{{ http.get(configUrl) | async | json}}</span>
\end{Code}
Here is a complete list of the built in pipes: \url{https://angular.io/api?type=pipe}.
\item Directives looks like html attributes. Behind each directive is a piece of code, which interacts with the underlaying DOM-elements, so basically directives can do anything. You can define your own directives using the \code{@Directive} class decoration, but that is beyond the scope of this course. Let's start by looking at some structural directives. By convention, all structural directives have names starting with *:
\begin{Code}
<div *ngIf="orders.length>0">Your Shopping Basket</div>

<ul>
  <li *ngFor="let salad of orders">{{salad.toString()}}</li>
</ul>
\end{Code}
There are more details to the NgFor, you can get more info from the structure you iteration over: 
\begin{Code}
*ngFor="let salad of orders; let i=index; let odd=odd; trackBy: trackById"
\end{Code}
For a complete explanation, please visit \\ \url{https://angular.io/guide/structural-directives}.

Other directives add functionality or style to a element, for example:
\begin{Code}
<input matInput name="food" placeholder="What to do">
\end{Code}
The \code{matInput} directive will tell angular material to style the \code{<input>} element. \code{name} and \code{placeholder} are standard html attributes associated with any \code{<input>} element. 

\section{Data binding}
The syntax to use for data binding depends on the direction the data flows.When data passes into a component, use the \code{[target]="expression"} syntax, for example:
\begin{Code}
<button [disabled]="isUnchanged">Save</button>
\end{Code}
The angular template is standard html, so if the input is static, you can write
\begin{Code}
<div class="special">Mental Model</div>
\end{Code}
Note, \code{"special"} is a string, you can not use embedded TypeScript expression here. 

When data flows from a component, for example a DOM event, use the\\ \code{(event)="expression"} syntax:
\begin{Code}
<button (click)="onSave($event)">Save</button>
\end{Code}
The source event can be found under the name \code{$event} in the expression string.

Angular support two way data binding, commonly used in forms:
\begin{Code}
<input name="todo" [(ngModel)]="form.todo">
\end{Code}
This works exactly as bound components in react, but you do not need to write the \code{onChange()} function. If you change the JavaScript object \code{form.todo}, angular will update the \code{<input>} value, and if the user writes in the form field, angular will update the \code{form.todo} object.

More about bindings can be found here:\\ \url{https://angular.io/guide/template-syntax#binding-syntax-an-overview}.
\section{@Input and @Output}
Components can have inputs, use the \code{@Input} decoration to bind a class attribute to a input:
\begin{Code}
@Component({
  selector: 'app-hello-world',
  template: `
    <h3>Hello World</h3>
    <p> {{text}}
  `
})
export class HeroChildComponent {
  @Input() text: string;
\end{Code}
\begin{Code}
<app-hello-world [text]="'Per was here!'"></app-hello-world>
\end{Code}

Output from a component is wrapped in an event emitted from the component, and follows the data binding pattern outlined above:
\begin{Code}
@Component({
    selector: 'app-child',
    template: `<button (click)="valueChanged()">Click me</button> `
})
export class AppChildComponent {
    @Output() valueChange = new EventEmitter();
    Counter = 0;
    valueChanged() { // You can give any function name
        this.counter = this.counter + 1;
        this.valueChange.emit(this.counter);
    }
}
\end{Code}
\begin{Code}
<app-child (valueChange)='displayCounter($event)'></app-child
\end{Code}

How data is passed in and out of components is described here: \url{https://angular.io/guide/component-interaction}.

\section{Service}
In angular components can interact trough services. Services enables bi-directional communication between any components in an angular app. A service is a JavaScript object which provides functionality to the components (functions they can call). Let's look an example, \code{SaladOrderService} which can mange orders of different salads:
\begin{Code}
import { Injectable } from '@angular/core';
import { Subject }    from 'rxjs';
 
@Injectable()
export class SaladOrderService {
  private orders = [];
  private orderSource = new Subject<Salad[]>();
  orders$ = this.orderSource.asObservable();
 
  orderSalad(salad: Salad) {
    this.orders.push(salad);
    this.saladSource.next(this.orders);
  }
}
\end{Code}
There is one function, \code{orderSalad()}, adding a salad to the order, and one RxJS Observable \code{orders$}, which can be used to listen to changes in the orders.

Any angular component can use a service simply by requesting a parameter of the service type in its constructor:
\begin{Code}
@Component({
  selector: 'app-compose-salad',
  template: `
    <h2>Compose Salad</h2>
    <form (ngSubmit)="submit()">...</form>
  `
})
export class ComposeSaladComponent {
  salad = {};  // will be filled by the html form
  constructor(private orderService: SaladOrderService) {  }
  submit() {
    this.orderService.orderSalad(this.salad);
  }
}
\end{Code}

\begin{Code}
@Component({
  selector: 'app-view-order',
  template: `
    <h2>Orders</h2>
    <ul>
      <li *ngFor="let salad of orders">{{salad.toString()}}</li>
    </ul>
  `
})
export class ViewOrderComponent {
  orders = [];
  constructor(private orderService: SaladOrderService) {
    orderService.orders$.subscribe(order => this.orders = orders);
  }
}
\end{Code}

\section{App.module.ts}
Angular needs to know the tag names of all angular components, and the types of all services. This is normally managed in the module of the application, the \texttt{app.moudle.ts} file:
\begin{Code}
@NgModule({
  declarations: [
    AppComponent,
    ViewOrderComponent,
    ComposeSaladComponent
  ],
  imports: [
    BrowserModule,
    BrowserAnimationsModule,
    FormsModule,

    MatButtonModule,
    MatCheckboxModule,
    MatFormFieldModule,
    MatInputModule,

    AppRoutingModule
  ],
  providers: [SaladOrderService],
  bootstrap: [AppComponent]
})
\end{Code}
\code{declarations} is a list of all components declared in this module. \code{providers} is a list of services declared in this module. \code{imports} is a list of other angular modules this module is depending on. After importing a module, you can use the components, pipes,  directives, and services declared in that module.
\end{itemize}

