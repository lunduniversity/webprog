%--- React Components ------------------------------------------------------------------------------
\section{Forms}
%--- Forms------------------------------------------------------------------------------
\begin{frame}[fragile] \frametitle{Forms}
HTML forms
\begin{itemize}
  \item browser adds user input to DOM elements
  \item user input is displayed
  \item render is not a pure function of \code{props} and state
\end{itemize}
\end{frame}

%--- Bound Component------------------------------------------------------------------------------
\begin{frame}[fragile] \frametitle{Bound Component}
\begin{itemize}
  \item component state is the ground truth
  \begin{itemize}
    \item user actions updates the component state
    \begin{itemize}
      \item event handlers update state
    \end{itemize}
    \item render is a pure function of state
  \end{itemize}
\end{itemize}
\end{frame}

%--- Bound Component example------------------------------------------------------------------------------
\begin{frame}[fragile] \frametitle{Bound Component Example}
\begin{CodeBox}{}
function BoundComponent() {
  const [text, setText] = useState("");
  function handleChange(event) {
    setText(event.target.value);
  }
  return (
    <form>
      <label htmlFor="text">Namn</label>
      <input type="text" value={text} onChange={handleChange} id="text"></input>
    </form>
  );
}
\end{CodeBox}
\end{frame}

%--- parametrised event handler------------------------------------------------------------------------------
\begin{frame}[fragile] \frametitle{Parametrised Event Handler}

\begin{CodeBox}{}
<input type="text" name="givenName" 
  value="this.state.givenName">
\end{CodeBox}

\begin{CodeBox}{}
const [state, setState] = useState({});
handleInputChange(event) {
  const target = event.target;
  const value =
    target.type === 'checkbox' ?
        target.checked : target.value;

  this.setState({...state, 
    [target.name]: value
  });
}
\end{CodeBox}
\end{frame}

%--- Unbound Component------------------------------------------------------------------------------
\begin{frame}[fragile] \frametitle{Unbound Component}
\begin{itemize}
  \item let the browser manage form state
  \begin{itemize}
    \item read the values from the DOM
    \item useRef() refers to the DOM element
  \end{itemize}
\end{itemize}
\end{frame}

%--- Unbound Component example------------------------------------------------------------------------------
\begin{frame}[fragile] \frametitle{Unbound Component Example}
\begin{CodeBox}{}
function UnboundComponent() {
  const inputRef = useRef(null);
  function handleSubmit(event) {
    event.preventDefault();
    alert(inputRef.current?.value);
  }
  return (
    <form onSubmit={handleSubmit}>
      <label htmlFor="text2">Unbound component</label>
      <input type="text" id="text2" ref={inputRef}></input>
    </form>
  );
}
\end{CodeBox}
\end{frame}

%--- Forms, file input------------------------------------------------------------------------------
\begin{frame}[fragile] \frametitle{Form, file input}
\begin{CodeBox}{}
<input type="file"/>
\end{CodeBox}
\vspace{8mm}

\begin{itemize}
  \item is read-only
  \item must be uncontrolled
\end{itemize}
\end{frame}

%--- React Hook Form ------------------------------------------------------------------------------
\begin{frame}[fragile] \frametitle{React Hook Form}

\begin{itemize}
  \item in this course we focus on the basic consepts
  \item use controlled forms during labs
  \item exam will  only cover controlled and uncontrolled forms
  \item feel free to use frameworks in the project
  \item React Hook Form is a popular framework
\end{itemize}
\end{frame}



