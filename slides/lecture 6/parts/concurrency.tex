%--- Concurrency------------------------------------------------------------------------------
\begin{frame}[fragile] \frametitle{Concurrency}
\begin{itemize}
  \item today concurrency is the key to performance
  \item parallellism is great for
  \begin{itemize}
    \item speed
    \item energy efficiency
  \end{itemize}
  \item but adds challenges for the programmer
  \begin{itemize}
    \item race conditions
    \item locks and deadlock
  \end{itemize}
\end{itemize}
\end{frame}

%--- Race Condition------------------------------------------------------------------------------
\begin{frame}[fragile] \frametitle{Race Condition}
\begin{itemize}
  \item the program behaviour depends on timing
  \item dependant concurrent activities
\end{itemize}
\vspace{3mm}

\begin{CodeBox}{}
let balance = 1000;
\end{CodeBox}
Lets do two transactions in parallel:
  \begin{columns}[onlytextwidth]
    \begin{column}{0.4\textwidth}
      \begin{CodeBox}{}
let tmp = balance;
\\ tmp is now 1000
tmp = tmp + 1000;
balance = tmp;
      \end{CodeBox}
    \end{column}
  \begin{column}{0.55\textwidth}
    \begin{CodeBox}{}
let tmp = balance;
\\ tmp is now 1000
tmp = tmp - 100;
balance = tmp;
      \end{CodeBox}
    \end{column}
    \begin{column}{0.3\textwidth}  \end{column}
  \end{columns}%
\vspace{3mm}
\code{balance} is either 2000 or 900. One transaction is lost.
\end{frame}

%--- Race Condition------------------------------------------------------------------------------
\begin{frame}[fragile] \frametitle{Race Condition}
Race condition can lead to:
\begin{itemize}
  \item lost transaction (previous slide)
  \item inconsistent state
\end{itemize}
\vspace{3mm}

\begin{CodeBox}{}
const seq = {cnt: 1, list: ['A ']};
\end{CodeBox}
Lets do two transactions in parallel:
  \begin{columns}[onlytextwidth]
    \begin{column}{0.4\textwidth}
      \begin{CodeBox}{}
seq. cnt++;



seq.list.push('B ');
      \end{CodeBox}
    \end{column}
  \begin{column}{0.55\textwidth}
    \begin{CodeBox}{}
for(let i=seq.cnt-1; i>=0; i++){
  console.log(seq.list[i]);
}
    \end{CodeBox}
    \end{column}
    \begin{column}{0.3\textwidth}  \end{column}
  \end{columns}%
\vspace{3mm}
Will print: \code{A undefined}
\end{frame}

%--- strategies for concurrency------------------------------------------------------------------------------
\begin{frame}[fragile] \frametitle{Strategies for Concurrency Problems}
Ignorance is bliss:
\begin{itemize}
  \item works surprisingly well
  \item conflicts are rare in many applications
  \item last write wins
\end{itemize}
\vspace{3mm}
Serialise access to shared data
\begin{itemize}
  \item locks and transactions: traditional approach for threads
  \item single thread
\end{itemize}
\vspace{3mm}
Functional approach:
\begin{itemize}
  \item pure functions
  \item immutable data structures
\end{itemize}
\end{frame}

%--- Locks------------------------------------------------------------------------------
\begin{frame}[fragile] \frametitle{Locks}
Not part of this course
\begin{itemize}
  \item locks are covered in: EDAP10 - Concurrent Programming
  \item to address this, locks was introduced
  \item \emph{semaphores} is the commonly used low level lock
  \item only one thread can own the lock
  \item all other needs to wait
  \item not a problem for performance, only a tiny part of the program is serialized
\end{itemize}
\end{frame}

%--- Deadlock------------------------------------------------------------------------------
\begin{frame}[fragile] \frametitle{Deadlock}
\begin{itemize}
  \item using locks can lead to deadlocks
  \item can only occur when:
  \begin{itemize}
    \item a thread holds a resource while requesting another
    \item there is a cycle in the dependency graph
  \end{itemize}
  \item you should know, in JavaScript:
  \begin{itemize}
    \item using \code{while} to wait for another function to change a variable leads to deadlock
  \end{itemize}
  \item deadlock analysis is covered in: EDAP10 - Concurrent Programming
\end{itemize}
\end{frame}

%--- Pure Functions and Immutable Data Structures------------------------------------------------------------------------------
\begin{frame}[fragile] \frametitle{Pure Functions and Immutable Data Structures}
Pure Function
\begin{itemize}
  \item output only depends on input
  \item stateless
\end{itemize}
\vspace{3mm}

Immutable Data Structures
\begin{itemize}
  \item data can not change over time
  \item ensure consistent data
  \item component state in react is based on this principle
\end{itemize}

\vspace{5mm}
Common in web frameworks, e.g. react component state
\end{frame}

%--- Polling and Busy Wait------------------------------------------------------------------------------
\begin{frame}[fragile] \frametitle{Polling and Busy Wait}
\begin{itemize}
  \item Polling:
  \begin{itemize}
    \item ask for a resource without locking
  \end{itemize}
  \item can be ok if it is unfrequent
  \item busy wait
  \begin{itemize}
    \item repeatedly ask for a resource in a loop that do nothing else
    \item  kills performance
    \item starves the other threads
    \item nothing happens
  \end{itemize}
\end{itemize}
\end{frame}

%--- JavaScript ------------------------------------------------------------------------------
\section{JavaScript}
%--- JavaScript------------------------------------------------------------------------------
\begin{frame}[fragile] \frametitle{JavaScript}
\begin{itemize}
  \item single threaded according to the specification
  \item the code runs from start to finish:
  \begin{itemize}
    \item can not be interrupted
    \item can not hand over execution to another function/thread
  \end{itemize}
  \item advantages:
  \begin{itemize}
    \item no need for locks
    \item inconsistent state can not occur (local state)
  \end{itemize}
  \item but:
  \begin{itemize}
    \item any longer computations blocks the entire application and GUI
    \item you should break down your app into small functions
    \item \code{alert()} is blocking.
    \item asynchronous events can cause race conditions (read/write data from server)
    \item lost transaction can occur
  \end{itemize}
\end{itemize}
\vspace{5mm}
\end{frame}

%--- Call Back Functions------------------------------------------------------------------------------
\begin{frame}[fragile] \frametitle{How do we wait for others to finish?}
Call Back Functions
\begin{itemize}
  \item no need for polling
  \item are central to JavaScript programming
  \item many APIs are based on call backs
  \item reactive programming: most of the application code are the call back functions
  \item called when events occur
  \item you can not control the order in which your call back functions are called
\end{itemize}
\end{frame}

%--- Current Thread Loop ------------------------------------------------------------------------------
\begin{frame}[fragile] \frametitle{Current Thread Loop}
Current Thread Loop manage the execution of callback functions
\begin{itemize}
  \item also called \emph{Event Loop}
  \item a queue of functions to execute
  \item functions in the queue are executed one by one, in sequence
  \item remember, JavaScript is single threaded and execution can not be interrupted
  \item DOM and network events and JavaScript code will add new call back functions to the execution queue
\end{itemize}
\end{frame}

%--- SetTimeout------------------------------------------------------------------------------
\begin{frame}[fragile] \frametitle{SetTimeout}
You can add your own functions to the execution queue:
\begin{itemize}
  \item \code{const id = setTimeout(foo, delay, arg1, arg2, ...)}
  \item \code{clearTimeout(id)}
  \item \code{const id = setInterval(foo, period, arg1, arg2, ...)}
  \item \code{clearInterval(id)}
  \item \code{this} defaults to \code{global}
  \item use a timeout of 0 to add the function directly to the execution queue
  \item do this to break long computations
\end{itemize}
\end{frame}

%--- SetTimeout------------------------------------------------------------------------------
\begin{frame}[fragile] \frametitle{SetTimeout}
\begin{CodeBox}{}
class MyTimer {
  counter = 0;
  tick() {
    console.log(this.counter++);
  }
}
const obj = new MyTimer();
setTimeout(obj.tick, 1000);
\end{CodeBox}
\vspace{4mm}
Use \code{bind} if you use \code{this}
\vspace{4mm}
\begin{CodeBox}{}
setInterval(obj.tick.bind(obj), 1000);
\end{CodeBox}

\end{frame}

%--- Execution Order ------------------------------------------------------------------------------
\begin{frame}[fragile] \frametitle{Execution Order}
Using callback functions, it can be hard to follow the order of execution.
\vspace{5mm}
\begin{CodeBox}{}
setTimeout(console.log, 1193, 'three');
setTimeout(console.log, 1058, 'four');
setTimeout(console.log, 1234, 'five');
setTimeout(console.log, 0, 'two');
console.log('one');
\end{CodeBox}
\end{frame}

%--- Pyramid of DOOM ------------------------------------------------------------------------------
\begin{frame}[fragile] \frametitle{Pyramid of DOOM}
Nesting callbacks are common, but leads to callback hell
\vspace{5mm}
\begin{CodeBox}{}
fetchFile(url, function(data1, error) {
  if (error) {
    handleError(error);
  } else {
    fetchFile(data1.nextUrl, function(data2, error) {
      if (error) {
        handleError(error);
      } else {
        fetchFile(data2.nextUrl, function(data3, error) {
          if (error) {
            handleError(error);
          } else {
            // ...continue after all files are loaded
          }
        });
      }
    })
  }
});\end{CodeBox}
\end{frame}
