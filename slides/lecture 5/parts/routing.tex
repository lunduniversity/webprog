%--- routing ------------------------------------------------------------------------------
\section{Routing}
%--- routing------------------------------------------------------------------------------
\begin{frame}[fragile] \frametitle{Routing}
\begin{itemize}
  \item the browser history is part of the user experience
  \item allows the user to navigate back to earlier visited pages
  \item an entry in the history is added when the user
  \begin{itemize}
    \item navigates to a new page using a link
    \item submits a form
  \end{itemize}
  \item traditionally, this loads a new page from the server
  \item when a new page is loaded, all JavaScript objects are lost
  \item singel page web application prevents this using \code{preventDefault()} on all relevant events
  \item only updating the DOM will impact the user experience:
  \begin{itemize}
    \item can not navigate using the browser history (back button)
    \item can not link to inner pages
  \end{itemize}
\end{itemize}
\end{frame}

%--- Routing Framework------------------------------------------------------------------------------
\begin{frame}[fragile] \frametitle{Routing Framework}
\begin{itemize}
  \item there is an API giving JavaScript direct access to the browser history
  \item using it manually is tedious and error prone
  \item let a router do the work for you
  \item a router have two main features: link and route
\end{itemize}

\vspace{10mm}
\begin{CodeBox}{}
npm install react-router-dom
\end{CodeBox}
\end{frame}

%--- Link ------------------------------------------------------------------------------
\section{Link}
%--- Link ------------------------------------------------------------------------------
\begin{frame}[fragile] \frametitle{Link}
Link component:
\begin{itemize}
  \item \code{<Link to="/about">About</Link>}
  \item appears and behaves as a normal link (\code{<a href='/about.html'>}):
  \begin{itemize}
    \item clicking on it will add an entry to the browser history
    \item this will update the url field in the browser
    \item the user can bookmark or copy any page in your application
    \item no new page is fetched from the server
    \item your JavaScript objects are untouched (preserve the application state)
  \end{itemize}
  \item add styling of active link using:\\ \code{<NavLink to="/react" activeClassName="hurray">}
\end{itemize}
\end{frame}

%--- Link Component------------------------------------------------------------------------------
\begin{frame}[fragile] \frametitle{Link Example}
\begin{CodeBox}{}
import { Link } from 'react-router-dom';
menu = (
  <nav>
    <Link to="/">Home</Link>
    <Link to="/users">Users</Link>
    <Link to="/contact">Contact</Link>
  </nav>
);
\end{CodeBox}
\end{frame}

%--- Route ------------------------------------------------------------------------------
\section{Route}
%--- Route ------------------------------------------------------------------------------
\begin{frame}[fragile] \frametitle{Route}
\code{<Route exact path="/">}
\begin{itemize}
  \item \code{path} - matches if the pattern is a prefix of the current browser url
  \item use \code{exact} for a complete match (no trailing characters)
  \item a \code{<Route>} with no \code{path} is always matched
  \item specify content using attributes:
  \begin{itemize}
    \item \code{component} - renders a react component if the \code{path} matches
    \item \code{render} - calls a function if the path matches, the function returns a react element (JSX expression)
    \item \code{children} - always rendered, a function returning a react element
  \end{itemize}
\end{itemize}
\end{frame}

%--- Route ------------------------------------------------------------------------------
\begin{frame}[fragile] \frametitle{Route}
\begin{CodeBox}{}
import { Route } from 'react-router-dom'
const routes = (
<div>
  <Route path="/" exact component={Home} />
  <Route path="/about"
           render={_ => <h1>About the app</h1>}>
</div>);
\end{CodeBox}
\end{frame}

%--- Switch ------------------------------------------------------------------------------
\begin{frame}[fragile] \frametitle{Switch}
\code{<Switch>}
\begin{itemize}
  \item renders the first \code{<Route>} that matches
  \item any other \code{<Route>}s will not be rendered
\end{itemize}

\vspace{5mm}
\begin{CodeBox}{}
import { Route, Switch } from 'react-router-dom'
const routes = (
<Switch>
  <Route path="/" exact component={Home} />
  <Route path="/items" component={Items} />
  <Route component={PageNotFound} />
</Switch>);
\end{CodeBox}
\end{frame}

%--- Path Parameters ------------------------------------------------------------------------------
\begin{frame}[fragile] \frametitle{Path Parameters}
the router pass data from the path to the component
\begin{itemize}
  \item specify parameters in the \code{path} using the syntax\code{:name}
  \item \code{props.match.params} is an object containing all path parameters
  \item or use the \code{useParams()}  hook to get that object
\end{itemize}

\vspace{5mm}
\begin{CodeBox}{}
const elem = <Route path="/item/:itemId" render={Items} />;

function Items(props) {
  let { itemId } = useParams();
  return <h1> Item {itemId} </h1>;
}
\end{CodeBox}
\end{frame}

%--- Hooks ------------------------------------------------------------------------------
\begin{frame}[fragile] \frametitle{Hooks}
\begin{itemize}
  \item introduced in react router 5.1
  \item can only be used in function components (stateless)
  \item can be used in any child of \code{Route}
  \item \code{useParams()} - returns the url path parameters
  \item \code{useRouteMatch()} - returns the path
\end{itemize}
\end{frame}

%--- Hooks ------------------------------------------------------------------------------
\begin{frame}[fragile] \frametitle{Hooks}
\begin{CodeBox}{}
function Item() {
      const { id } = useParams();
      return <p>Item {id}</p>;
}
function App() {
  return (
      <Router>
        <Route path="item/:id">
          <Item />
        </Route>
      </Router>
  );
}
\end{CodeBox}
\end{frame}

%--- Nested Routes ------------------------------------------------------------------------------
\begin{frame}[fragile] \frametitle{Nested Routes}
\begin{itemize}
  \item \code{<Route>} are react components and can be nested
  \item \code{path} is absolute
  \item \code{useRouteMatch()} helps you build relative paths
\end{itemize}
\begin{CodeBox}{}
function About() {
  let { path, url } = useRouteMatch();
  return (
      <Switch>
        <Route exact path={path} component={GeneralInfo} />
        <Route path={path + "/copyright"} 
                 component={Copyright} />
      </Switch>);
  );
}
\end{CodeBox}
\end{frame}

%--- Router ------------------------------------------------------------------------------
\begin{frame}[fragile] \frametitle{Router}

\begin{itemize}
  \item any web application starts with a html file \code{index.html} 
  \item the html file includes the react JavaScript code
  \item all \code{<Route>}s must be inside a router
\end{itemize}

\code{<BrowserRouter>}
\begin{itemize}
  \item \code{http://domain.se/item/42}
  \item to allow direct links: the server must return the main html file for all urls
  \item node do this for you
  \item apache can be configured with redirect
\end{itemize}

\code{<HashRouter>}
\begin{itemize}
  \item \code{http://domain.se/#/item/42}
  \item compatible with all servers
\end{itemize}
\end{frame}

