\section{HTML}

%---------------------------------------------------------------------------------
\begin{frame}[fragile]
\frametitle{HTML}
\begin{lstlisting}[style=htmlcssjs]
<!DOCTYPE html>
<html>
  <head>
    <meta charset="utf-8">
    <title>Hello World</title>
    <link rel="stylesheet" href="css/styles.css">
    <script src="my-awsome-code.js"></script>
    <base  href="https://www.cs.lth.se/eda095/">
  </head>

  <body>
    <h1>Hello World</h1>
    <p>This is my page. It is awesome.
  </body>
</html>
\end{lstlisting}
\end{frame}

%---------------------------------------------------------------------------------
\begin{frame}[fragile]
\frametitle{HTML - element}

Semantic tags\\
\html{<h1>, <h2>, <p>,  <abbr>, <code>, <samp>, <kbd>, <var>, <footer>, <header>, <details>, <nav>}\ldots
\bigskip

Structure\\
\html{<table>, <ul>, <ol>, <div>, <span>}\ldots
\bigskip

Functionality included\\
\html{<form>, <input>, <select>, <button>, <a>}\ldots
\bigskip
\color{structure}

Learn more about HTML tags \url{https://developer.mozilla.org/en-US/docs/Web/HTML/Element}
\url{https://www.w3schools.com/tags/default.asp}
\end{frame}

%---------------------------------------------------------------------------------
\begin{frame}[fragile]
\frametitle{HTML - elements}
\color{structure}
\begin{itemize}\color{structure}
  \item syntax:
  \begin{itemize} 
    \item \html{<tag-name attr1="v1" attr2="v2">content</tag-name>}
  \end{itemize}
  \item \html{</end-tag>}
  \begin{itemize} 
    \item optional for some tags
    \item required for some tags
  \end{itemize}
  \item content:
  \begin{itemize} 
    \item sequence of text and HTML-elements
    \item rendered on the page
  \end{itemize}
  \item proper nesting among elements is required
\end{itemize}

\end{frame}

%---------------------------------------------------------------------------------
\begin{frame}[fragile]
\frametitle{HTML - attributes}
\color{structure}

\html{<a href="http://cs.lth.se">my link</a>} 

\begin{itemize}\color{structure}
  \item always text, use double quotation marks
  \item semantics depends on tag-name
  \item common/special attributes:
  \begin{itemize}\color{structure}
    \item \html{id} - optional attribute, unique on a page\\ can be used to find/refer to an element
    \item \html{for} - refer to another element, uses the \html{id}
    \item \html{name} - reference in some context, for example in \html{<form>}
    \item \html{tabindex} - makes the element focusable
    \item \html{aria-*} - accessibility, aid for screen readers, when no other textual representation exists
    \item \html{style} - used for styling
    \item \html{class} - used for styling (do not relate to JavaScript classes)
  \end{itemize}
\end{itemize}
\end{frame}

%---------------------------------------------------------------------------------
\begin{frame}[fragile]
\frametitle{HTML}
\begin{lstlisting}[style=htmlcssjs]
<form>
  <div>
    <label for="email-id">Email address</label>
    <input type="email"  name="email-name" id="email-id" placeholder="name@example.com">
    <p>We'll never share your email with anyone else.
  </div>
  <div>
    <label for="password">Password</label>
    <input type="password" name="password" id="password">
  </div>
  <div>
    <input type="checkbox" id="check">
    <label for="check">remember me</label>
  </div>
  <button type="submit">Submit</button>
</form>
\end{lstlisting}
\end{frame}
