\documentclass[aspectratio=1610]{beamer}
\usepackage{iftex}
\usepackage{pdfpages}
\ifLuaTeX\else
	\usepackage[utf8]{inputenc}
	\usepackage[T1]{fontenc}
\fi
\usetheme[
% titleimagecolor=red,       % [gray], darkgray, red, blue, green
% titleimagemargin=2mm,      % Distance [2mm]    Frame around title page image
% navigationsymbols=false,   % true   / [false]  Navigation symbols in the foot
% mathseriffont=false,       % true   / [false]  Serif / non-serif math fonts
% foot=true,                 % [true] / false    Footline or not
% nofootslidenum=false       % true   / [false]  Keep slide num even when foot=false
% footlogo=true,             % [true] / false    Put LU logo to the left of footer
% english=true,              % [true] / false    English / Swedish logo
% LTHlogo=false,             % true   / [false]  Use LTH logo instead of LU on title and end pages.
% blackenumeratenumber=true, % [true] / false    Black enumerate numbers, o.w. Lund bronze
% blackitemmark=false,       % true   / [false]  Black item marks, o.w. Lund bronze
% defaultfont=palatino,      % [palatino], beamer, lu
% sectionframe=true,
]{ulund}
%%%%%%%%%%%%%%%%%%%%% Layout commands 
%%%% Foot
% \ulundfootleft{\insertshortauthor}
% \ulundfootmid{\insertshorttitle}
% \ulundfootright{\insertframenumber}% {\insertframenumber:\inserttotalframenumber}
%%%% Titleimage
% \titleimage{Pictures/ULUNDcolor} % Replaces the LU image. Voids option titleimagecolor
%%%%%%%%%%%%%%%%%%%%%%%%%%%%%%%%%%%

\newcommand{\code}[1]{\lstinline{#1}}
\title[JavaScript]{JavaScript}
\author[EDAF90]{%
  EDAF90 Web Programming\newline
  Per Andersson}
%%%%%%%%%%%%%%%%%%%%%
%\usepackage{verbatim}
%%%%%%%%%%%%% Verbatim code box
\usepackage[skins,listings]{tcolorbox}
\ifLuaTeX\else
	\tcbuselibrary{listingsutf8}
\fi
\newtcblisting{CodeBox}[2][]{% Only code
  colframe=black,
  colback=white,
  arc=1pt,
  boxrule=0.5pt,
  top=0mm,bottom=0pt,left=0pt,
  colbacktitle=gray!40,
  coltitle=black,
  fonttitle=\sffamily,
  listing only,
  hbox,
  listing options={
    basicstyle=\small\ttfamily,
    breaklines=true,
    columns=fullflexible,
    language=Java,
    basicstyle = \ttfamily,
    showstringspaces=false,
  },
  title=#2,#1
}

\lstset{
    basicstyle=\small\ttfamily,
    breaklines=true,
    columns=fullflexible,
    language=Java,
    basicstyle = \ttfamily,
    showstringspaces=false,
    morekeywords= {undefined, NaN, function, export, var, let, of, in}
}

%%%%%%%%%%%%%%%%%%%%%
%%%%%%%%%%%%%%%%%%%%%
%%%%%%%%%%%%%%%%%%%%%

\begin{document}
\begin{frame}[plain]% Use plain to suppress footline box
  \titlepage
\end{frame}

%%%%%%%%%%%%%%%
\begin{frame}
  \frametitle{JavaScript}

\begin{itemize}
\item ``the world's most misunderstood programming language''
\item working name at Netscape 1995: \emph{LiveScript}
\item syntax and name in common with Java
\item object function oriented language
\item interpreted
\item dynamically typed 
\item run in any web browser and node.js
\end{itemize}
\end{frame}

%%%%%%%%%%%%%%%
\begin{frame}
  \frametitle{Interpreted}

\begin{itemize}
\item no compilation $\rightarrow$ no compilation errors
\item design decision: try to execute everything
  \begin{itemize}
  \item many silent errors
  \item weird and unexpected behaviour
  \item for example a miss spelled  property name
  \end{itemize}
\item the programmer have more responsibility
\item programmers needs extended language knowledge
\item use jslint to check your code
\item write test cases to catch compilation errors
\end{itemize}
\end{frame}


%---Types ---------------------------------------------------
%---Types ---------------------------------------------------
\begin{frame}[fragile]
  \frametitle{Types}
Seven data types that are primitives (immutable):
\begin{itemize}
  \item  \verb|boolean|
  \item  \verb|null| - \verb|typeof null| $\rightarrow$ \verb|object|
  \item  \verb|undefined|
  \item  \verb|number|
  \item  \verb|bigint| - literal syntax: \verb|42n|
  \item  \verb|string|
  \item  \verb|symbol| - unique and immutable
\end{itemize}
All other values have the type:
\begin{itemize}
  \item  \verb|object|
\end{itemize}
\end{frame}

%---Dynamically Typed Language ---------------------------------------------------
\begin{frame}[fragile]
  \frametitle{Dynamically Typed Language}
  JavaScript is dynamically typed.\\
  A variable have a value and a type. Both can change.\\
  This includes objects. You can add and remove properties.\\
  \vspace{5mm}
  \begin{columns}[onlytextwidth]
    \begin{column}{0.4\textwidth}
      \begin{CodeBox}{ valid JavaScript}
let a = 'Per';  
a = 0;
a = null;
a = undefined;
      \end{CodeBox}
    \end{column}
  \begin{column}{0.55\textwidth}
    \begin{CodeBox}{typeof}
typeof 'Per' === 'string';
typeof 0 === 'number';
typeof null === 'object';
typeof undefined = 'undefined';
      \end{CodeBox}
    \end{column}
    \begin{column}{0.3\textwidth}  \end{column}
  \end{columns}%
  \vspace{5mm}
\verb|typeof| returns a string. The value is one of the 8 types.
\end{frame}

%---Type Conversion ---------------------------------------------------
\begin{frame}[fragile]
  \frametitle{Type Conversion}
  Enforce type conversion with expressions.\\
  \begin{CodeBox}{ type converting expressions}
typeof (+'42') // 'number'
typeof (!!null) // 'boolean'
  \end{CodeBox}
  \vspace{5mm}
  Use type converting functions: \verb|Number()|, \verb|String()|, and \verb|Boolean()|.
  \begin{CodeBox}{ type converting expressions}
typeof Number('42') // 'number'
typeof Number('Per') // 'number'
typeof Boolean('false') // 'boolean'
typeof String(42) // 'string'
  \end{CodeBox}
\end{frame}

%---Automatic Type Conversion ---------------------------------------------------
\begin{frame}[fragile]
  \frametitle{Automatic Type Conversion}
  JavaScript will automatically convert values when needed.\\
  There is a strong preference to convert to \verb|string|.\\
  The type conversion algorithm have some non intuitive side effect. This is the root of some of the \emph{bad parts}
   of JavaScript.
\vspace{5mm}
  \begin{CodeBox}{ automatic type conversion}
3 + '42';      // '342'
null + 'Per'; //  'nullPer'
3 == '3'       // true
  \end{CodeBox}
\end{frame}

%---Strings ---------------------------------------------------
\begin{frame}[fragile]
  \frametitle{Strings}
String literals and templates
\begin{itemize}
  \item \verb|'single quotation mark'|
  \item \verb|"double quotation mark"|
  \item \verb|`string templates| \\
           \verb|can span multiple lines| \\
           \verb|and contain embedded expressions: 1+2=${1+2}`|
\end{itemize}
Operations
\begin{itemize}
  \item \verb|'Per'+ ' ' + 'Andersson'|
  \item \verb|'Per'.lenght|
  \item \verb|'Per'.toUpperCase()|
  \item \verb|'Per'[0]|
\end{itemize}
\begin{columns}[onlytextwidth]
  \begin{column}{0.3\textwidth}  \end{column}
  \begin{column}{0.4\textwidth}
    \begin{alertblock}{Note}strings are immutable \end{alertblock}
  \end{column}
  \begin{column}{0.3\textwidth}  \end{column}
\end{columns}%
 \end{frame}


%---Truthy/Falsy ---------------------------------------------------
%---Truthy/Falsy ---------------------------------------------------
\begin{frame}[fragile]
  \frametitle{Truthy/Falsy}
  Falsy:
  \begin{itemize}
    \item \verb|false|
    \item \verb|0|
    \item \verb|0n|
    \item \verb|""|, \verb|''|, \verb|``|
    \item \verb|null|
    \item \verb|undefined|
    \item \verb|NaN|
  \end{itemize}
\begin{CodeBox}{no need for}
if (name === null || name.length === 0){
  name = 'anonymous';
}
\end{CodeBox}
\end{frame}

%---Short Circuit ---------------------------------------------------
\begin{frame}[fragile]
  \frametitle{Short Circuit}
  Logic operations return the value of one operand.\\
  Nullish coalescing operator (??), right hand side iff LHS is \code{null} or \code{undefined}
\begin{columns}[onlytextwidth]
  \begin{column}{0.51\textwidth}
\begin{CodeBox}{ some expressions}
  a = 'Per' || 'default value';
  b = '' || 'default value';
  c = 'Per' || null;
  d = NaN || undefined;
  
  e = 'Per' && 'Andersson';
  f = undefined && 'Andersson';
  g = 'Per' && NaN;
  h = ref && ref.value;

  i = '' ?? 'default value';
\end{CodeBox}
  \end{column}
  \begin{column}{0.05\textwidth}  \end{column}
  \begin{column}{0.48\textwidth}
\begin{CodeBox}{evaluates to}
  a = 'Per';
  b = 'default value';
  c = 'Per';
  d = undefined;
  
  e = 'Andersson';
  f = undefined;
  g = NaN;
  h = ref ? ref.value : ref;

  i = '';
\end{CodeBox}
  \end{column}
\end{columns}%
\end{frame}

%---Optional Chaining operator ---------------------------------------------------
\begin{frame}[fragile]
  \frametitle{Optional Chaining operator}
  \begin{itemize}
    \item \code{object?.property}
    \item access a property or calls a function
    \item short-circuit and return \code{undefined} if:
    \begin{itemize}
      \item \code{object} is \code{null} or \code{undefined}, or
      \item \code{property} is not a property of \code{object}
    \end{itemize}
  \end{itemize}
\begin{CodeBox}{throws no exceptions}
  function myFunction(obj) {
    console.log( obj.?prop );
    console.log( obj.?[1]);
    console.log( obj.func.?());
    obj.func = 3;
    console.log( obj.func.?());
    // Uncaught TypeError: obj.func is not a function
    obj?.a?.b?.[0]?.()?.c;
  }
\end{CodeBox}

\end{frame}


%---Equality and sameness ---------------------------------------------------
\begin{frame}[fragile]
  \frametitle{Equality and sameness}
  There are four equality algorithms in ES2015:
  \begin{itemize}
    \item Abstract/Loose Equality: \verb|==|, \verb|!=|
    \begin{itemize}
      \item triggers type conversion leading to unexpected behaviour
    \end{itemize}
    \item Strict Equality: \verb|===|, \verb|!==|, compare type and value
    \begin{itemize}
      \item conform to IEEE 754 (so \code{NaN != NaN}, and \code{-0 == +0)}
    \end{itemize}
    \item \verb|Object.is()|: Same Value, as strict equality except for {NaN, -0,} and \code{+0}
  \end{itemize}
\begin{columns}[onlytextwidth]
  \begin{column}{0.45\textwidth}
\begin{CodeBox}{evaluates to true}
  1 == '1';
  [1, 2] == '1,2';
  [1, 2] != '1, 2';
  'true' != true;
\end{CodeBox}
  \end{column}
  \begin{column}{0.45\textwidth}
\begin{CodeBox}{evaluates to true}
  -0 === +0;
  0 == false
  1 !== '1';
  null == undefined;
  null !== undefined;
\end{CodeBox}
  \end{column}
  \begin{column}{0.3\textwidth}  \end{column}
\end{columns}%
Check out the \href{https://dorey.github.io/JavaScript-Equality-Table/}{JavaScript Equality Table} 
\end{frame}


%---Functions ---------------------------------------------------
%---Functions ---------------------------------------------------
\begin{frame}[fragile] \frametitle{Functions}

\begin{itemize}
  \item functions are values - \code{Function} objects
  \begin{itemize}
    \item assign functions to variables
    \item use functions as arguments to other functions
  \end{itemize}
  \item call by value - like in Java (objects are references)
  \item default return value:
  \begin{itemize}
    \item \code{undefined}
    \item \code{this} in constructors
  \end{itemize}
  \item three ways to create functions:
  \begin{itemize}
    \item function declaration
    \item function expression
    \item \code{Function} constructor (not recommended)
  \end{itemize}
\end{itemize}

\end{frame}

%---Function Declaration ---------------------------------------------------
\begin{frame}[fragile] \frametitle{Function Declaration}

\begin{itemize}
  \item is a statement
  \item no need to use semicolon after a function declaration
  \item creates
  \begin{itemize}
    \item a \code{Function} object
    \item a variable with the function name
  \end{itemize}
\end{itemize}
\begin{CodeBox}{function declaration}
function calcRectArea(width, height) {
  return width * height;
}

console.log(calcRectArea(5, 6));
\end{CodeBox}
\end{frame}

%---Function Expression ---------------------------------------------------
\begin{frame}[fragile] \frametitle{Function Expression}

\begin{itemize}
  \item is an expression
  \item creates a \code{Function} object
  \item the function name is optional, omitting it creates an anonymous function
  \item the name is stored in the \code{Function} object, can only be used inside the function
  \item you must store the value to use the function
\end{itemize}
\begin{CodeBox}{function expression}
let calcRectArea = function foo(width, height) {
  return width * height;
}

console.log(calcRectArea(5, 6));
\end{CodeBox}
\end{frame}

%---Default Parameters ---------------------------------------------------
\begin{frame}[fragile] \frametitle{Default Parameters}
\begin{itemize}
  \item function parameters default to \code{undefined}
  \item parameters can have other default values (ES2015)
  \item parameters are available to later default parameters
  \item default parameters are evaluated at call time
\end{itemize}

\begin{CodeBox}{rest parameters}
function multiply(a, b = 1) {
  return a * b;
}

function greet(name, 
                    greeting,
                    message = greeting + ' ' + name) {
    return [name, greeting, message];
}
\end{CodeBox}
\end{frame}

%---Rest Parameters ---------------------------------------------------
\begin{frame}[fragile] \frametitle{Rest Parameters}
\begin{itemize}
  \item must be the last named parameter
  \item all remaining arguments are wrapped into an \code{Array}
\end{itemize}
\vspace{5mm}

\begin{CodeBox}{rest parameters}
function sloppySum(first, ...theRest) {
  return theRest.reduce((previous, current) => {
    return previous + current;
  });
}
\end{CodeBox}
\end{frame}

%---Arguments Object ---------------------------------------------------
\begin{frame}[fragile] \frametitle{Arguments Object}
\begin{itemize}
  \item \code{arguments} is an \code{Array}-like object
  \item contains all arguments
  \item doesn't have \code{Array}'s built-in methods like \code{forEach()} and \code{map()}
  \item properties
  \begin{itemize}
    \item \code{arguments.callee}
    \item \code{arguments.caller}
    \item \code{arguments.length}
    \item \code{arguments[@@iterator]}
  \end{itemize}
\end{itemize}

\begin{CodeBox}{arguments}
function foo(a, b, c) {
  console.log(arguments[1]);
}
foo(1, 2, 3);
\end{CodeBox}
\end{frame}

%--- Arrow Function ---------------------------------------------------
\begin{frame}[fragile] \frametitle{Arrow Function}

\begin{itemize}
  \item convenient syntax
  \item is an expression
  \item creates an anonymous function
  \item without own bindings to the \code{this}, \code{arguments}, \code{super}, or \code{new.target}
  \item these values are retained from enclosing lexical context
  \item ill suited as methods, and they cannot be used as constructors
\end{itemize}
\begin{CodeBox}{syntax}
([param[, param]]) => {
   statements
}

param => expression
\end{CodeBox}
\end{frame}


%--- Arrow Function 2 ---------------------------------------------------
\begin{frame}[fragile] \frametitle{Arrow Function, examples}

\begin{CodeBox}{example of arrow functions}
let sqr = x => x*x;

let calcRectArea = (width, height) => width * height;

let pi = _ => Math.PI;

let myLogger = (msg) => {
  console.log(new Date() + ': ' + msg); 
};

let foo = (width, height) => { width * height };
\end{CodeBox}
\end{frame}

%--- Function Oriented Programming ---------------------------------------------------
\begin{frame}[fragile] \frametitle{Function Oriented Programming}

JavaScript have all features of a function oriented language.
%\vspace{8mm}

\begin{CodeBox}{function oriented programming}
let list = [1, 2, 3, 4, 5];
let a = list.map(x => x + 2);
let b = a.filter((x) => x % 2 === 0);
b.forEach(x => console.log(x));
let c = b.reduce((sum, x) => sum + x, 0);
\end{CodeBox}
\begin{CodeBox}{chaining}
let sum = [1, 2, 3, 4, 5];
sum.map(x => x + 2)
.filter((x) => x % 2 === 0)
.reduce((sum, x) => sum + x, 0);
\end{CodeBox}
\end{frame}

%--- Closure ---------------------------------------------------
\begin{frame}[fragile] \frametitle{Closure}

\begin{itemize}
  \item lexical scope
  \item a closure gives you access to an outer function’s scope from an inner function
  \item closures are created every time a function is created, at function creation time
\end{itemize}
\vspace{5mm}

\begin{CodeBox}{closure}
let name = 'Per Andersson';
let foo = function() {
  name = 'anonymous';
}
console.log(name);
foo();
console.log(name);
\end{CodeBox}
\end{frame}

%--- Closure 2 ---------------------------------------------------
\begin{frame}[fragile] \frametitle{Closure}

\begin{itemize}
  \item remember, functions are values.
  \item inner functions can be returned from a function.
\end{itemize}

\vspace{4mm}

\begin{CodeBox}{closure}
let foo = function() {
  let cnt = 0;
  return _ => cnt++;
}

let idGenerator = foo();

console.log(idGenerator());
some_async_function(idGenerator);
another_async_function(idGenerator);
\end{CodeBox}
\end{frame}



%---scope ---------------------------------------------------
%---Variables ---------------------------------------------------
\begin{frame}[fragile] \frametitle{Variables}
JavaScript separate variable names and property names.
\begin{itemize}
 \item to find a property, you must first find the object in a variable
 \item \code{this.someProperty}
\end{itemize}
\vspace{3mm}
Global name space
\begin{itemize}
  \item shared by all functions and modules
  \item high risk of name conflict
  \item do not use
\end{itemize}
\vspace{3mm}
Shared name space for:
\begin{itemize}
  \item  global variables
  \item parameters
  \item local variables
\end{itemize}
\end{frame}

%---Scope ---------------------------------------------------
\begin{frame}[fragile] \frametitle{Scope}
Two different kind of scopes:
\begin{itemize}
  \item  function scope
  \begin{itemize}
    \item \code{var}
  \end{itemize}
  \item block scope (ES2015)
  \begin{itemize}
    \item \code{let}
    \item \code{const}
    \item works like scope in Java
  \end{itemize}
\end{itemize}
\vspace{4mm}
\begin{itemize}
  \item reading an undeclared name throws a \code{ReferenceError}
  \item assigning to an undeclared name creates it as a global variable
\end{itemize}

\end{frame}

%---Function Scope ---------------------------------------------------
\begin{frame}[fragile] \frametitle{Function Scope}
\begin{itemize}
  \item declare variables using \code{var}
  \item the scope is the current execution context
  \begin{itemize}
    \item the function
    \item the global context
  \end{itemize}
  \item redeclaration of names are allowed
  \item considered bad practice today
\end{itemize}
\end{frame}

%---Function Scope 2 ---------------------------------------------------
\begin{frame}[fragile] \frametitle{Function Scope, example 1}

\begin{CodeBox}{}
function foo() {
  y = 1; // Throws a ReferenceError in strict mode.
  var x = 3;
  if (true) {
    var x = 2;
  }
  return x;
}
console.log(y); // undefined
console.log(foo()); // 2
console.log(y); // 1
\end{CodeBox}
\end{frame}

%---Function Scope 3 ---------------------------------------------------
\begin{frame}[fragile] \frametitle{Function Scope, example 2}

\begin{CodeBox}{}
function foo() {
  for (var i=0; i<2; i++) {
    for (var i=0; i<2; i++) {
      console.log(i);
    }
  }
  return x;
}
foo()  // 0, 1
\end{CodeBox}
\end{frame}

%---Function Scope 4 ---------------------------------------------------
\begin{frame}[fragile] \frametitle{Function Scope, example 3}

\begin{CodeBox}{}
function foo() {
  for (var i=0; i<2; i++) {
    setTimeout(function(){ console.log(i); }, 0);
  }
}
foo()  // 2, 2
\end{CodeBox}
\end{frame}

%---Name Hoisting ---------------------------------------------------
\begin{frame}[fragile] \frametitle{Name Hoisting}
In function scope
\begin{itemize}
  \item all declared variables are created before any code is executed
  \item name declarations are lifted to top of function
  \item initialisation remain in place
  \item function expressions are not hoisted
\end{itemize}

\begin{CodeBox}{}
function foo() {
  console.log(x);  // undefined
  var x = 3;
  console.log(x);  // 3
}
\end{CodeBox}
\end{frame}

%---Immediately Invoked Function Expressions ---------------------------------------------------
\begin{frame}[fragile] \frametitle{Immediately Invoked Function Expressions}
\begin{itemize}
  \item creates a closure when declared
  \item if inside a loop
  \begin{itemize}
    \item one closure for each loop iteration
    \item variables are initialised in each loop iteration
    \item clone and freeze outer statte (loop variable)
  \end{itemize}
\end{itemize}

\begin{CodeBox}{}
// outer scope
var x = 2;
(function() {
  // inner hidden scope
  var x = 3;
})();

console.log(x);
// more outer scope
\end{CodeBox}
\end{frame}

%---Immediately Invoked Function Expressions 2 ---------------------------------------------------
\begin{frame}[fragile] \frametitle{IIFE, example 2}

\begin{CodeBox}{}
var domElements = [...];
for (var i=0; i<10; i++) {
  (function () {
    let id = i;
    domElement[i].onClick(
      _ => { alert('clicked ' + id); }
    );
  })();  
}
\end{CodeBox}
\end{frame}



%---modules ---------------------------------------------------
%---JavaScript modules ---------------------------------------------------
\begin{frame}[fragile] \frametitle{JavaScript modules}
  Introduced in ES6
\begin{columns}[onlytextwidth]
  \begin{column}{1\textwidth}
\begin{CodeBox}{my-module.js}
function cube(x) {
  return x * x * x;
}
const foo = Math.PI + Math.SQRT2;
const text = "private in module";
export { cube, foo };
\end{CodeBox}
\begin{CodeBox}{some-code.js}
import { cube, foo } from './my-module.js';

console.log(cube(3));
console.log(foo);
\end{CodeBox}
  \end{column}
\end{columns}%
\end{frame}

%--- CommonJS modules---------------------------------------------------
\begin{frame}[fragile] \frametitle{CommonJS modules}
  Common in environments not supporting JavaScript Modules, for example node.
\begin{columns}[onlytextwidth]
  \begin{column}{1\textwidth}
\begin{CodeBox}{my-module.js}
function cube(x) {
  return x * x * x;
}
const foo = Math.PI + Math.SQRT2;
\end{CodeBox}
\begin{CodeBox}{some-code.js}
const stuff = require('./my-module.js');

console.log(stuff.cube(3));
console.log(stuff.foo);
\end{CodeBox}
  \end{column}
\end{columns}%
\end{frame}


%---objects ---------------------------------------------------
%---Objects ---------------------------------------------------
\begin{frame}[fragile] \frametitle{Objects}
\begin{itemize}
  \item an object is a map: string $\rightarrow$ any value
  \item objects have properties - a (string, value) par in the map (attributes, methods)
  \item properties can have any name, including reserved words and operations
  \item access properties using:
  \begin{itemize}
    \item dot notation: \code{myObj.prop}
    \item array index notation: \code{myObj['prop']}
  \end{itemize}
  \item \code{typeof objRef === 'object'}
  \item add properties by writing to them \code{myObj.newProp = 'adding stuff';}
  \item remove properties by: \code{delete myObj.newProp}
\end{itemize}
\end{frame}

%---Objects, example 1 ---------------------------------------------------
\begin{frame}[fragile] \frametitle{Objects, example 1}
 \begin{CodeBox}{basic object}
let myObject = {
  givenName: 'Per',
  familyName: 'Andersson',
  selector: 'givenName',
  getValue: function () {
    return this[this.selector];
  }
}
 
console.log(myObject.getValue());
myObject.selector = 'familyName';
console.log(myObject.getValue());
 \end{CodeBox}
\end{frame}

%---Object Literals ---------------------------------------------------
\begin{frame}[fragile] \frametitle{Object Literals}
\begin{itemize}
  \item superset of JSON
  \item comma separated list of properties inside \code{\{ \}}
  \item a property is defined by: \code{property-name : value}
  \item name in plain text, quotes if needed 
  \item value is any JavaScript expression
  \item \code{\{a:a\}} is the same as \code{\{a\}}
\end{itemize}
\begin{CodeBox}{object literal}
let myObject = {
  five: 2 + 3,
  '5': 'five',
  '+' : 'plus',
  'null': 'not a good idea name'
 }
\end{CodeBox}
\end{frame}

%---Object Literals 2 ---------------------------------------------------
\begin{frame}[fragile] \frametitle{Object Literals}
\begin{itemize}
  \item object literals are cheap
  \item use them frequently
  \item they bring structure and readability to programs
\end{itemize}
\begin{CodeBox}{object literals}
let myPoints = [{a: 0, y: 0}, {x:10, y:15}];

function foo(a, b, c, d, e, f) {
 console.log('d = '+ d);
}
function bar(arg) {
 console.log('d = '+ arg.d);
}
\end{CodeBox}
\end{frame}

%--- Named Parameters ---------------------------------------------------
\begin{frame}[fragile] \frametitle{Named Parameters}
Remember, \code{foo} and \code{bar} prints parameter \code{d}.
\vspace{5mm}
\begin{CodeBox}{What is printed?}
foo(0, 0, 0, 0, 1, 1, 1);
bar({a: 0, b: 0, c: 0, d: 1, e: 1, f:1});
\end{CodeBox}
\vspace{10mm}
Did you notice thet foo have one extra parameter compared to the arguments list?
\\ Too few, or extra parameters do not raise errors in JavaScript. 
\end{frame}

%--- Constructor Functions ---------------------------------------------------
\begin{frame}[fragile] \frametitle{Constructor Functions}

\begin{itemize}
  \item like classes in Java: forms objects when they are created
  \item normal functions
  \item the intended usage differs
  \item by convention: use leading capital letter in name (similar to class names in Java)
  \item only call constructor functions using \code{new}:
  \begin{itemize}
    \item new creates an empty object and binds it to \code{this}
    \item the constructor function is called, adds properties to the object
    \item the return value of the constructor function will be the result of \code{new}
    \item remember: the default return value of functions called by \code{new} is \code{this}
  \end{itemize}
  \item combine with closure and you have the power
\end{itemize}
\end{frame}

%--- Constructor Functions  Example ---------------------------------------------------
\begin{frame}[fragile] \frametitle{Constructor Function Example}

\begin{columns}[onlytextwidth]
  \begin{column}{0.5\textwidth}
\begin{CodeBox}{class definition}
function Point(x, y) {
  this.x = x || 0;
  this.y = y || 0;
  this.getX = function() {
    return this.x;
  }
}
\end{CodeBox}
  \end{column}
  \begin{column}{0.5\textwidth}
\begin{CodeBox}{create instances}
let point1 = new Point(3, 6);
let point2 = new Point();
let point2 = new Point(5);
let point3 = 
  new Point(null, 5);
\end{CodeBox}
  \end{column}
\end{columns}%
\end{frame}

%--- this---------------------------------------------------
\begin{frame}[fragile] \frametitle{this}
\begin{itemize}
  \item \code{this} is defined in all functions
  \item arrow functions, \code{this} from the enclosing scope is used
  \item its value depends on how the function is called:
  \begin{itemize}
    \item function call: \code{foo()} - the global object
    \item dot notation: \code{obj.foo()} - the object left of the dot
    \item explicit: \code{Function.prototype.call()}
    \item explicit: \code{Function.prototype.bind()} - creates a new function with a predefined value for \code{this}
    \item as an DOM event handler - the element the event fired from (not all cases for all browsers)
    \item as an inline DOM event handler - the DOM element on which the listener is placed
  \end{itemize}
\end{itemize}
\end{frame}

%--- self---------------------------------------------------
\begin{frame}[fragile] \frametitle{self}
When a function is a ``class method''
\begin{itemize}
  \item you do not know if \code{this} refers to the right object
  \item use closure to fix this
  \item or use arrow functions
\end{itemize}
\begin{CodeBox}{}
function Person() {
  var self = this; // Some choose `that` instead of `self`. 
                   // Choose one and be consistent.
  self.age = 0;

  setInterval(function growUp() { self.age++; }, 1000);
  setInterval(() => { this.age++;  }, 1000);
}
\end{CodeBox}
\end{frame}
%---Undefined Names ---------------------------------------------------
\begin{frame}[fragile] \frametitle{Access to Undefined Names}
Variables and properties have distinct name spaces.
\\ \vspace{4mm}
Variables:
\begin{itemize}
  \item read: throws ReferenceError
  \item write: creates a variable in the global scope
\end{itemize}
\vspace{5mm}
Properties:
\begin{itemize}
  \item read: evaluates to \code{undefined}
  \item write: adds the property to the object
\end{itemize}
\end{frame}

%--- Prototype based Inheritance ---------------------------------------------------
\begin{frame}[fragile] \frametitle{Prototype Based Inheritance}

\begin{itemize}
  \item all object inherit from another object
  \item objects forms a \emph{prototype chain}
  \item property name lookup follows the prototype chain
  \item you can access the prototype chain (but don't):
  \begin{itemize}
    \item \code{Object.getPrototypeOf(object)}
    \item \code{Object.setPrototypeOf(object, chain)}
  \end{itemize}
  \item the chain ends with \code{null}
\end{itemize}
\end{frame}

%--- Set up Prototype Chain ---------------------------------------------------
\begin{frame}[fragile] \frametitle{Set up Prototype Chain}
Setting up the prototype chain when a new object is created:
\begin{itemize}
  \item you can do it manually (but don't)
  \item \code{new} do the work for you
  \item all \code{Function} objects have a property \code{prototype}
  \item remember, constructor functions are instance of \code{Function}
  \item \code{new}:
  \begin{itemize}
    \item creates an empty object
    \item {\bf and} set its parent in the prototype chain to the \code{prototype} of the constructor function
  \end{itemize}
  \item all names in the \code{prototype} of the constructor function are now in the prototype chain of the new object
\end{itemize}
\end{frame}

%--- Set up Prototype Chain, example ---------------------------------------------------
\begin{frame}[fragile] \frametitle{example}
\begin{CodeBox}{prototype chain}
function Family(name) {
  this.givenName = name || 'Per';
}
Family.prototype.familyName = 'Andersson';
Family.prototype.toString = function() {
  return this.givenName + ' ' + this.familyName;
}

let me = new Family();
let sister = new Family('Anette');
console.log(me + ' and my sister ' + sister);
me.familyName = 'Andersson Nilsson';
console.log(me + ' and my sister ' + sister);
\end{CodeBox}
\end{frame}

%--- Property Name Lookup ---------------------------------------------------
\begin{frame}[fragile] \frametitle{Property Name Lookup}
Property read:
\begin{itemize}
  \item follows the prototype chain
  \item return the first value found
  \item return \code{undefined} if the end of the prototype chain is reached
\end{itemize}
\vspace{8mm}
Property write:
\begin{itemize}
  \item do not follows the prototype chain
  \item writes to the referenced object (left hand side of the dot)
  \item update if the name existed
  \item adds the property if the name did not exist
\end{itemize}
\end{frame}

%--- Inheritance ---------------------------------------------------
\begin{frame}[fragile] \frametitle{Inheritance}
\begin{itemize}
  \item the \code{prototype}s of the constructor functions form the prototype chain
  \item \code{Object.create()} creates an object with a given prototype chain
  \item use it to create the \code{prototype} of the constructor function
  \item explicit call the constructor of the superclass
\end{itemize}
\vspace{2mm}
\begin{CodeBox}{JuniorFamily extends Family}
function JuniorFamily(name) {
  Family.call(this);
}
JuniorFamily.prototype = Object.create(Family.prototype);
JuniorFamily.prototype.toString = function() {
  return this.givenName + ' ' + 
             this.familyName + '  jr.';
}
\end{CodeBox}
\end{frame}

%--- Class ---------------------------------------------------
\begin{frame}[fragile] \frametitle{Class}
a "Java class" corrsponds to two objects in JavaScript
\begin{itemize}
  \item a constructor function:
  \begin{itemize}
    \item its name is part of the variable name space
    \item place static stuff here
  \end{itemize}
  \item a prototype object
  \begin{itemize}
    \item the object to add to the prototype chain
    \item place any stuff to be inherited by the instances here
  \end{itemize}
\end{itemize}
\vspace{5mm}

\code{Class} was introduced in ECMAScript 2015
\begin{itemize}
  \item syntactical sugar, set up the prototype chin as outlined above
  \item \code{static} will add the property to the constructor function object
\end{itemize}
\end{frame}
%--- Class Example ---------------------------------------------------
\begin{frame}[fragile] \frametitle{Class Example}
\begin{CodeBox}{}
class Family {
  constructor(name) {
    super();
    this.givenName = name || 'Per';
    Family.count = Family.count + 1;
  }
  toString() {
    return this.givenName + ' ' + this.familyName;
  }
  familyName = 'Andersson';
  static count = 0;
}
\end{CodeBox}
\end{frame}

%--- Class Extends---------------------------------------------------
\begin{frame}[fragile] \frametitle{Class Extends}
\begin{itemize}
  \item the constructor in a derived class must call \code{super()} before you can use \code{this}
  \item you can: \code{extend null}
\end{itemize}
\begin{CodeBox}{}
class JuniorFamily extends Family {
  constructor(name) {
    super(name);
  }
 toString() {
   return this.givenName + ' ' + 
              this.familyName + '  jr.';
  }
}
\end{CodeBox}
\end{frame}

%--- Standard Classes---------------------------------------------------
\begin{frame}[fragile] \frametitle{Standard Classes}
In JavaScript there are many standard classes. Some important: 
\begin{itemize}
  \item \code{Object} - default base class for all objects
  \item \code{Function extends Object} - base class for all functions
  \item \code{Array} - base class for array litterals
\end{itemize}
\end{frame}

%--- More to learn---------------------------------------------------
\begin{frame}[fragile] \frametitle{More to learn}

This is the basics of objects in JavaScript, but there are more under the surface\ldots
\vspace{8mm}
\begin{CodeBox}{}
Object.defineProperty(obj, "prop", {
    value: "test",
    writable: false
});
\end{CodeBox}
\vspace{8mm}
This is however out of scope for this course.
\end{frame}



%---arrays ---------------------------------------------------
%--- Arrays---------------------------------------------------
\begin{frame}[fragile] \frametitle{Arrays}
\begin{itemize}
  \item variable size and type
  \item index must be number
  \item \code{myArray['per'] = 3;} - adds a property to the array object
\end{itemize}
\end{frame}

%---Destructuring assignment ---------------------------------------------------
%--- Destructuring assignment---------------------------------------------------
\begin{frame}[fragile] \frametitle{Destructuring assignment}
\begin{itemize}
  \item unpack arrays and objects
  \item use:
  \begin{itemize}
    \item left hand side of assignment
    \item function parameters
  \end{itemize}
  \item can have default values
  \item can be nested
  \item the tail of an array can be stored in a variable: \code{...remainingValues}
\end{itemize}
\begin{CodeBox}{}
const foo = ['red', 'green'];
const [one, two, three = 'blue'] = foo;
console.log(one); // "red"
console.log(three); // "default three"
const [one, ...rest] = foo;
\end{CodeBox}
\end{frame}

%--- Destructuring assignment, Objects---------------------------------------------------
\begin{frame}[fragile] \frametitle{Destructuring assignment}
\begin{CodeBox}{}
const user = {
  id: 42,
  displayName: 'jdoe',
  fullName: {
    firstName: 'John',
    lastName: 'Doe'
  }
};

const {id:selectedId} = user;

function whoIs({displayName, fullName: {firstName: name}}) {
  return `${displayName} is ${name}`;
}
\end{CodeBox}
\end{frame}

%--- Spread Syntax---------------------------------------------------
\begin{frame}[fragile] \frametitle{Spread Syntax}
The spread syntax \code{...} can be used on \\
An iIterable, such as an array or string, can be expanded instead of:
\begin{itemize}
  \item zero or more arguments (for function calls)
  \item elements (for array literals)
\end{itemize}

An object expression to be expanded instead of
\begin{itemize}
  \item  zero or more key-value pairs (for object literals)
\end{itemize}

\begin{CodeBox}{}
function sum(x, y, z) {
  return x + y + z;
}

const numbers = [1, 2, 3];

const total = sum(...numbers);
\end{CodeBox}
\end{frame}

%--- Spread Syntax 2 ---------------------------------------------------
\begin{frame}[fragile] \frametitle{Spread Syntax}
\begin{CodeBox}{}
const parts = ['shoulders', 'knees']; 
const lyrics = ['head', ...parts, 'and', 'toes']; 

const obj1 = { foo: 'bar', x: 42 };
const obj2 = { foo: 'baz', y: 13 };

const clonedObj = { ...obj1 };

const augmentedObj = { ...obj1, name: 'Per' };

const mergedObj = { ...obj1, ...obj2 };

\end{CodeBox}
\end{frame}



%--- Automatic Semicolon Insertion ---------------------------------------------------
\begin{frame}[fragile]
  \frametitle{Automatic Semicolon Insertion}
  If needed, a semicolon is added at the end of a line.
\begin{CodeBox}{the function will return undefined}
  function() { return
  1; }
\end{CodeBox}

\vspace{4mm}
Common to use minify to minimise script download size. All white spaces are removed.
\begin{columns}[onlytextwidth]
  \begin{column}{0.35\textwidth}
\begin{CodeBox}{works}
  let myVar = 9
  if (myVar === 9) {
  }
\end{CodeBox}
  \end{column}
  \begin{column}{0.65\textwidth}
\begin{CodeBox}{syntax error after minify}
  var myVar = 9 if (myVar === 9) {}
\end{CodeBox}
  \end{column}
  \begin{column}{0.3\textwidth}  \end{column}
\end{columns}%

\vspace{4mm}
Use jslint to detect these problems.
\end{frame}

%---Strict mode ---------------------------------------------------
\begin{frame}[fragile]
  \frametitle{Strict mode}
Converting mistakes into errors.

\begin{CodeBox}{Whole-script strict mode syntax}
'use strict';
var v = "Hi! I'm a strict mode script!";
\end{CodeBox}

\begin{CodeBox}{Function-level strict mode syntax}
function strict() {
  'use strict';
  function nested() { return 'And so am I!'; }
  return "Hi!  I'm a strict mode function!  " + nested();
}
function notStrict() { return "I'm not strict."; }
\end{CodeBox}


\end{frame}


%%%%%%%%%%%
\end{document}
