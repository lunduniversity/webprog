\documentclass[aspectratio=1610]{beamer}
\usepackage{iftex}
\usepackage{pdfpages}
\ifLuaTeX\else
	\usepackage[utf8]{inputenc}
	\usepackage[T1]{fontenc}
\fi
%%%%%%%
% \usepackage{layout}
% \usepackage{lipsum}
%%%%%%%
\usetheme[% Complete settings. Default value in []
% titleimagecolor=red,       % [gray], darkgray, red, blue, green
% titleimagemargin=2mm,      % Distance [2mm]    Frame around title page image
% navigationsymbols=false,   % true   / [false]  Navigation symbols in the foot
% mathseriffont=false,       % true   / [false]  Serif / non-serif math fonts
% foot=true,                 % [true] / false    Footline or not
% nofootslidenum=false       % true   / [false]  Keep slide num even when foot=false
% footlogo=true,             % [true] / false    Put LU logo to the left of footer
english=false,              % [true] / false    English / Swedish logo
% LTHlogo=false,             % true   / [false]  Use LTH logo instead of LU on title and end pages.
% blackenumeratenumber=true, % [true] / false    Black enumerate numbers, o.w. Lund bronze
% blackitemmark=false,       % true   / [false]  Black item marks, o.w. Lund bronze
% defaultfont=palatino,      % [palatino], beamer, lu
% sectionframe=true,
]{ulund}
%%%%%%%%%%%%%%%%%%%%% Layout commands 
%%%% Foot
% \ulundfootleft{\insertshortauthor}
% \ulundfootmid{\insertshorttitle}
% \ulundfootright{\insertframenumber}% {\insertframenumber:\inserttotalframenumber}
%%%% Titleimage
% \titleimage{Pictures/ULUNDcolor} % Replaces the LU image. Voids option titleimagecolor
%%%%%%%%%%%%%%%%%%%%%%%%%%%%%%%%%%%
\title[EDAF90 Webbprogrammering]{EDAF90 - \\Webbprogrammering}
\author[Per Andersson]{%
  Per Andersson\newline
  Dept.\@ of CS, Lund University}
%%%%%%%%%%%%%%%%%%%%%
%\usepackage{verbatim}
%%%%%%%%%%%%% Verbatim code box
\usepackage[skins,listings]{tcolorbox}
\ifLuaTeX\else
	\tcbuselibrary{listingsutf8}
\fi
\lstdefinestyle{CodeStyle}{basicstyle = \ttfamily}
\newtcblisting{CodeBox}[2][]{% Only code
  colframe=black,
  colback=white,
  arc=1pt,
  boxrule=0.5pt,
  top=0mm,bottom=0pt,left=0pt,
  colbacktitle=gray!40,
  coltitle=black,
  fonttitle=\sffamily,
  listing only,
  title=#2,#1}
%%%%%%%%%%%%%%%%%%%%%
%%%%%%%%%%%%%%%%%%%%%
%%%%%%%%%%%%%%%%%%%%%
\begin{document}
\begin{frame}[plain]% Use plain to suppress footline box
  \titlepage
\end{frame}

%%%%%%%%%%%%%%%
\begin{frame}
  \frametitle{Course Content}

Kursen handlar om tekniken för program som kör i en webbläsare:
\begin{itemize}
\item fördjupad programmeringsfärdigheter
\item grundläggande asynkron programming and strömmar
\item en introduktion till kapplöpning och dödläge (concurrency)
\item webbläsare som exekveringsplattform: webb formulär, händelsehantering, schemaläggning.
\item grundläggande terminologi och koncept inom webbprogrammering
\item grundläggande principer för HTLM, CSS
\item ramverk för webbapplikationer (react, angular, bootstrap)
\item nätverksprotokoll för internet
\end{itemize}

\end{frame}


%%%%%%%%%%%%%%%
\begin{frame}
  \frametitle{Kursen kommer inte lära dig}
\begin{itemize}
\item avancerad eller "best practice" för HTML
\item avancerad eller "best practice" för CSS
\item GUI design/interaktions design/användarpespektiv
\item säkerhet
\item serverprogrammering
\item mjukvaruarkitekturer och design patterns för webben
\item DevOps
\end{itemize}

\end{frame}

%%%%%%%%%%%%%%%
\begin{frame}
  \frametitle{Kompletterande kurser}
\begin{itemize}
\item EITF05 - Webbsäkerhet
\item EITN41 - Avancerad Webbsäkerhet
\item MAMN25 - Interaction Design
\item EDAP10  - Flertrådad programmering
\item EDAF75 - Databasteknik
\item EDAN40 - Funktionsprogrammering
\item EDAG05 - Agil programvaruutveckling - projekt
\end{itemize}

\end{frame}

%%%%%%%%%%%%%%%
\begin{frame}
  \frametitle{Struktur}
Kursen består av:
\begin{itemize}
  \item 10 föreläsningar
  \item 4 labbar
  \item 1 projekt
\end{itemize}
\vspace{2mm}
Labbarna görs i grupper om 2:
\begin{itemize}
 \item lös uppgiften tillsammans
 \item lös uppgiften självständigt, innan redovisning:
   \begin{enumerate}
     \item jämför lösningarna
     \item alternera vems lösning som redovisas
   \end{enumerate}
\end{itemize}
\end{frame}

%%%%%%%%%%%%%%%
\begin{frame}
  \frametitle{CEQ 2021}
Positivt:
\begin{itemize}
  \item rolig kurs
  \item bra och lärorika labbar
  \item bra med fritt projekt, styr inte uppgiften
  \item projektet ska behållas även om det ibland upplevs som stressigt
\end{itemize}
\vspace{2mm}
Kritik:
\begin{itemize}
  \item otydliga instruktioner till labbarna, kod behöver skrivas om mellan labbar
  \item mycket text att läsa, 3 böcker
  \item oklart vad kursen omfattar
  \item bloggar och hemsidor gör det oklart vad kursen omfattar
  \item svår med flera nya språk: JavaScript, HTML, JSX, Angular template language
  \item fler övningar
  \item hög arbetsbelastning
\end{itemize}
\end{frame}

%%%%%%%%%%%%%%%
\begin{frame}
  \frametitle{CEQ 2021}
Åtgärder:
\begin{itemize}
  \item labbinstruktionerna skrivs om
  \item labbarna blir mer styrda
  \item reflektionsfrågor i labbarna
  \item träning i att läsa bloggar och hemsidor behövs för att följa ämnet
  \item all text ska inte läsas, du behöver göra en egen bedömning
  \item vad som ingår i kursen står i förorden till \emph{lecture notes}
\end{itemize}
\end{frame}

%%%%%%%%%%%%%%%
\begin{frame}
  \frametitle{Registrering}
\begin{itemize}
  \item kursregistrering i LADOK
  \item anmäl dig till labbarna i \href{https://sam.cs.lth.se/Labs}{SAM}
\end{itemize}

\end{frame}

%%%%%%%%%%%%%%%
\begin{frame}
  \frametitle{Kursmaterial}

Kursmaterialet består av böcker, bloggar och annat material online, se läsinstruktionerna i canvas.
\\ \vspace{3mm}
Böcker
\begin{itemize}
\item \href{https://github.com/getify/You-Dont-Know-JS}{You Dont Know JS by Kyle Simpson} 
\item \href{https://eloquentjavascript.net/}{Eloquent JavaScript by Marijn Haverbeke}
\item \href{http://shop.oreilly.com/product/9780596517748.do}{JavaScript: The Good Parts by Douglas Crockford} (för den intresserade)
\end{itemize}

Annat material
\begin{itemize}
\item \href{http://fileadmin.cs.lth.se/cs/Education/EDAF90/lectures/lecture-notes.pdf}{Lecture notes for EDAF90}
\item \href{https://developer.mozilla.org/en-US/docs/Learn}{Mozilla online material: tutorials and language reference}
\item fler länkar finns under läsanvisningarna inför varje föreläsning.
\end{itemize}

\end{frame}

%%%%%%%%%%%% End frame
\begin{frame}[plain]
  \endpage
\end{frame}

%%%%%%%%%%%
\end{document}
